\title{Reference documentation for ESMValTool}
\date{\today}

\documentclass[12pt]{article}
\usepackage{graphicx}
\usepackage[square,comma,numbers,sort&compress]{natbib}
\usepackage[pdfauthor={Martin.Evaldsson@smhi.se},
            pdftitle={Overview of ESMValTool},
            pdfkeywords={EMBRACE, work package 4},
            pdfsubject={EMBRACE WP 4 visualisation tool}]{hyperref}
\usepackage[hyphenbreaks]{breakurl}
\usepackage[all]{hypcap}
\usepackage{url}
\usepackage{varioref}
\usepackage{multirow}
\usepackage{rotating}
\usepackage{fancyvrb}
\usepackage{color}
\usepackage{amsmath}
\usepackage{titlesec}

\definecolor{dark-red}{rgb}{0.4,0.15,0.15}
\definecolor{dark-blue}{rgb}{0.15,0.15,0.4}
\definecolor{medium-blue}{rgb}{0,0,0.5}
\hypersetup{
        colorlinks, linkcolor={dark-red},
        citecolor={dark-blue}, urlcolor={medium-blue}
}

\newenvironment{myverb}{\footnotesize\begin{Verbatim}[frame=single, fontsize=\footnotesize]}{\end{Verbatim}}
%% Define a new 'leo' style for the package that will use a smaller font.
\makeatletter
\def\url@leostyle{%
  \@ifundefined{selectfont}{\def\UrlFont{\sf}}{\def\UrlFont{\small\ttfamily}}}
\makeatother
%% Now actually use the newly defined style.
\urlstyle{leo}

\newcommand{\docref}[1]{`\emph{#1}'}
\newcommand{\xmltag}[1]{\texttt{$<$#1$>$}}

\setcounter{secnumdepth}{5}

\begin{document}
\maketitle
\clearpage
\tableofcontents
\clearpage


%%%%%%%%%%%%%%%%%%%%%%%%%%%%%%%%%%%
% 
% XML CONFIGURATION FILE TAGS
% 
%%%%%%%%%%%%%%%%%%%%%%%%%%%%%%%%%%%
\phantomsection
\section{XML-configuration file}\label{section:XML_configuration_file}
The XML-configuration file has the overall structure, 
\begin{Verbatim}[frame=single, fontsize=\footnotesize]
<namelist>
    <namelist_summary>
        A description of the current namelist
    </namelist_summary>
    <GLOBAL>
        <key1> value1 </key1>
        <key2> value2 </key2>
        <key3 attr="attr_value> value3 </key3>
        ...
    </GLOBAL>

    <MODELS>
        <model> entries to match data set 1 </model>
        <model> entries to match data set 2 </model>
        ...
    </MODELS>

    <DIAGNOSTICS>
        <diag>
            <description> Textual description of diagnostic</description>
            <variable_def_dir>  path  </variable_def_dir>
            <variable> variable1 </variable>
            <variable> variable2 </variable>
            ...
            <field_type> field_type1  </field_type>
            <field_type> field_type2  </field_type>

            <plot_script_cfg_dir> path  </plot_script_cfg_dir>

            <plot_script cfg="cfg_filename">  plot_script1.suffix  
                                                        </plot_script>
            <plot_script cfg="cfg_filename">  plot_script2.suffix
                                                        </plot_script>
            ...
            <model> diag specific data set 1  </model>
            <model> diag specific data set 2  </model>
            ...
        </diag>

        <diag>
            ...
        </diag>
        ...
    </DIAGNOSTICS>
</session>
\end{Verbatim}
\normalsize


%%%%%%%%%%%%%%%%%%%%%%%%%%%%%%%%%%%
% <namelist_summary>-TAG
%%%%%%%%%%%%%%%%%%%%%%%%%%%%%%%%%%%
\phantomsection
\subsection{\texorpdfstring{\xmltag{namelist\_summary}}-tag}
To be written

%%%%%%%%%%%%%%%%%%%%%%%%%%%%%%%%%%%
% <GLOBAL>-TAG
%%%%%%%%%%%%%%%%%%%%%%%%%%%%%%%%%%%
\phantomsection
\subsection{\texorpdfstring{\xmltag{GLOBAL}}-tag}
To be written

%%%%%%%%%%%%%%%%%%%%%%%%%%%%%%%%%%%
% <MODELS>-TAG
%%%%%%%%%%%%%%%%%%%%%%%%%%%%%%%%%%%
\phantomsection
\subsection{\texorpdfstring{\xmltag{MODELS}}-tag}
To be written

%%%%%%%%%%%%%%%%%%%%%%%%%%%%%%%%%%%
% <DIAGNOSTICS>-TAG
%%%%%%%%%%%%%%%%%%%%%%%%%%%%%%%%%%%
\phantomsection
\subsection{\texorpdfstring{\xmltag{DIAGNOSTICS}}-tag}
To be written

%%%%%%%%%%%%%%%%%%%%%%%%%%%%%%%%%%%
% <diag>-TAG
%%%%%%%%%%%%%%%%%%%%%%%%%%%%%%%%%%%
\phantomsection
\subsubsection{\texorpdfstring{\xmltag{diag}}-tag}
To be written
% ############ XML CONFIGURATION FILE TAGS #################



%%%%%%%%%%%%%%%%%%%%%%%%%%%%%%%%%%%
% 
% AUXILIARY PLOT SCRIPT FUNCTIONS
% 
%%%%%%%%%%%%%%%%%%%%%%%%%%%%%%%%%%%
\phantomsection
\section{Plot script auxiliary functions}\label{section:plot_script_aux}
To ensure some consistency across various plot scripts, which may be
written in different languages altogether, this section recommends a number
of ``standard''-routines. The routines need to be implemented for each
plot script language in use.

%%%%%%%%%%%%%%%%%%%%%%%%%%%%%%%%%%%
% LIST OF AUX ROUTINES
%%%%%%%%%%%%%%%%%%%%%%%%%%%%%%%%%%%
\phantomsection
\subsection{List of auxiliary routines/practices}
\subsubsection{Standard output handling}
The python wrapper running a plot script collects its standard
output/error, and, at the end of the plot script execution, writes
this information back to standard output. From the user side the
verbosity of this output is controlled by the \xmltag{verbosity}-tag
in the namelist, an integer ranging from zero and upwards. A higher
value indicates more verbose output. On the plot script side this is
implemented as an `info\_output(\ldots)'-function. Below is a
minimalist NCL-version of this function, other script
languages have similar functions implemented. 
\begin{Verbatim}
procedure info_output(output_string [1] : string,\
                      verbosity [1] : integer,\
                      required_verbosity [1] : integer)
begin
    if (verbosity .ge. required_verbosity) then
        print("info: " + output_string)
    end if
end
\end{Verbatim}
The purpose of this function is two-fold, to filter out messages with
low priority (=high `required\_verbosity'), and to tag the message as
an info message (to set it apart from warnings and errors).

% ERROR HANDLING
\phantomsection
\subsubsection{Error handling}
Because not all script languages used in ESMValTool propagates errors
properly a proxy mechanism for error handling is in place. As the plot
script exectution comes to an end the standard output/error  is
scanned for certain keywords indicating an error/warning was
encountered during the execution. For NCL, these are, 
\begin{itemize}
\item `\texttt{fatal:}' - indicates an error
\item `\texttt{error:}' - indicates an error
\item `\texttt{warning:}' - indicates a warning
\end{itemize}

If any of the error keywords are present, the script wrapper will
will dump all output to standard output and raise an
exception. Thus, to explicitly raise an error from an
NCL plot script-routine it is necessary to use the construct, 
\begin{Verbatim}
           print("fatal: NCL-error message")
\end{Verbatim}
Note that whether a \texttt{warning:} should halt execution is
controlled via a switch in the xml-namelist (see above). Also note
that most, but not all, native NCL-routines uses the prefix
\texttt{fatal:} to indicate a critical error. If an error is
suspected, but not catched by the \texttt{fatal:}/\texttt{error:}
scan, it could be because a native NCL routine does not follow the
\texttt{fatal:} convention, but prints different error message
instead. In such cases, try increasing the verbosity such that all
the plot script output is printed and check for any error manually.

The keywords for indicating errors/warnings are defined on a source
code level, hence it is possible to use different keywords for other
plot script languages.

% OUTFILE TYPE
\phantomsection
\subsubsection{Type of output}
Many scripting languages supports different output formats, e.g.,
postscript, pdf, png, etc.. The user can indicate the preferred output
type via the \xmltag{output\_file\_type} in the namelist. Below shows
the NCL-snippet handling this request in an NCL plot script,
\begin{Verbatim}
    file_type = getenv("ESMValTool_output_file_type")
    if(ismissing(file_type)) then
        file_type = "PS"  ; Default output type
    end if
\end{Verbatim}
Note that not all script languages supports all output types

% CONSISTENT FIGURE FILENAME OUTPUT
\phantomsection
\subsubsection{Figure filenames}
When the output from a plot script is a figure, this function provides
a consitent naming of the output figure files. The main motivation for
this routine is to simplify the search for output figures, or a
specific set of output figures by following a naming convention. below
is the declaration for the NCL version of this function.
\begin{Verbatim}
   output_filename = get_figure_outfile_name(plot_type, \
                                             variable, \
                                             field_number,\
                                             aux_title_info, \
                                             idx_mod)
\end{Verbatim}


%%%%%%%%%%%%%%%%%%%%%%%%%%%%%%%%%%%
% PLOT SCRIPT TEMPLATE
%%%%%%%%%%%%%%%%%%%%%%%%%%%%%%%%%%%
\phantomsection
\subsection{Plot script template}
Use the existing plot scripts and the \texttt{plot\_scripts/MyDiag.ncl}
as templates for creating new plot diagnostics. 

% ############ PLOT SCRIPT AUXILIARY FUNCTIONS #################



%%%%%%%%%%%%%%%%%%%%%%%%%%%%%%%%%%%
% 
% ESMVALTOOL CONTROL FLOW
% 
%%%%%%%%%%%%%%%%%%%%%%%%%%%%%%%%%%%
\phantomsection
\section{ESMValTool control flow}\label{section:control_flow}
To be written

% ############ ESMVALTOOL CONTROL FLOW #################



%%%%%%%%%%%%%%%%%%%%%%%%%%%%%%%%%%%
% 
% ESMVALTOOL DATA FLOW
% 
%%%%%%%%%%%%%%%%%%%%%%%%%%%%%%%%%%%
\phantomsection
\section{ESMValTool data flow}\label{section:data_flow}
To be written

% ############ ESMVALTOOL DATA FLOW #################



%%%%%%%%%%%%%%%%%%%%%%%%%%%%%%%%%%%
% 
% OVERVIEW OF AVAILABLE PLOT_TYPES
% 
%%%%%%%%%%%%%%%%%%%%%%%%%%%%%%%%%%%
\phantomsection
\section{Overview of available plot scripts}\label{section:overview_plot_types}
To be written. Currently this information is only available on the
ESMValTool wiki page\cite{ESMValTool_wiki}.

% ############ OVERVIEW OF AVAILABLE PLOT_TYPE #################




% ############ ESMVALTOOL CONCEPTS #################

%%%%%%%%%%%%%%%%%%%%%%%%%%%%%%%%%%%%
%
% DOCUMENTATION AND CODE CONVENTIONS
%
%%%%%%%%%%%%%%%%%%%%%%%%%%%%%%%%%%%%
\phantomsection
\section{ESMValTool documentation and code conventions}\label{section:code_conventions}
This section proposes a recommended documentation and code convention
practice within the ESMValTool. 

\phantomsection
\subsection{Documentation of diagnostics}
New diagnostics are implemented in branches of the subversion
repository\cite{ESMValTool_repo}. As part of the process of
reintegrating them to trunk, they should be documented on the
projected wiki\cite{ESMValTool_wiki}. This doucmentation should
contain a technical description of the diagnostics, example of output
(figures and/ord netCDF), relevant references if applicable,
instructions on how to run the plot script. See existing plot script
documentation for further details.

\phantomsection
\subsection{Inline documentation of scripts}
\subsubsection{Python}
The recommended standard is PEP257\cite{pep257:2001}.

\phantomsection
\subsubsection{NCL}
The recommended standard is to follow the inline documentation used
for the NCL functions in the online
NCL-documentation\cite{ncar-ncl-homepage}

\phantomsection
\subsubsection{R}
The recommended standard is to follow the inline documentation used
for the R functions at CRAN.

\phantomsection
\subsection{Code convention guidelines}
The better part of the python routines in ESMValTool have been
formatted according the Python PEP8 style\cite{pep8:2001}. The
PEP8-style comes with a script for style guide checking, this script
is available in a repository branch, 

\begin{sloppypar}
\url{https://svn.dlr.de/ESM-Diagnostic/source/branches/EMBRACE/pep8-compliance/util/pep8-checker/}
\end{sloppypar}

Additional python resources used for formatting code is the
\texttt{redindent.py}-tool, available in the util folder of the same
branch as is given above, and
\texttt{pyflakes}\cite{PyFlakes}.

\phantomsection
\subsubsection{NCL}
The better part of the NCL functions have been formatted according to
the Python standard PEP8\cite{pep8:2001}. A modified PEP8-checker,
taking some of the NCL specific requirements into account, is
available at, 

\begin{sloppypar}
\url{https://svn.dlr.de/ESM-Diagnostic/source/branches/EMBRACE/pep8-compliance/util/ncl-checker/}
\end{sloppypar}

The NCL-version is adaption of the Pyhton checker and works
satisfactorily as long as one keeps in mind the false positives it
finds due to language differences between Python and NCL. These false
positives may be addressed in the future depending on priorities.

For consistent indentation across NCL files the emacs lisp script
\texttt{ncl.el}\footnote{\url{http://www.ncl.ucar.edu/Applications/Files/ncl.el}},
but with the default indent set to four instead of two (consistent
with Python, see~\cite{pep8:2001}). Using \texttt{ncl.el} requires
comments to be written as `\texttt{;;}' to not be re-indented (which
usually is not desirable). Note that \texttt{ncl.el} can be run via
emacs batch mode and is therefore equally accessible for both emacs
and non-ecmacs users.

\phantomsection
\subsubsection{R}
The implemented R diagnostics have been reformatted using the "R
parser tree"\cite{Tidying-R-code}. Note that this method can only be
considered to semi-automatic since it does preserve comments (they
need to be repatched) and does not produce very nice line breaks.

% ############ DOCUMENTATION AND CODE CONVENTIONS #################


%%%%%%%%%%%%%%%%%%%
% TESTING
%%%%%%%%%%%%%%%%%%%
\phantomsection
\section{Testing}\label{section:testing}
For testing purposes a Robot Framework\cite{Robot-Framework}
configuration file has been setup. The defined tests are on a namelist
level, i.e., each namelist in the test case file is executed and the
framework decides whether the test is passed or not depending on the
existance and size of the expected output files from that namelist
(figure/netCDF). Since running the tests requires namelists consistent
with the platform at hand an individual setup is required for each
platform. See the, 

\begin{Verbatim}[frame=single, fontsize=\footnotesize]
nml/test_suites/smhi/testcase_namelists.txt
\end{Verbatim}
file in the repository\cite{ESMValTool_repo} for an example of a test
case file. Running the test suite is done by, 

\begin{Verbatim}[frame=single, fontsize=\footnotesize]
pybot nml/test_suites/smhi/testcase_namelists.txt
\end{Verbatim}
% ############ TESTING #################


%%%%%%%%%%%%%%%%%%%%%%%%%%%%%%%%%%%
% 
% BIBLIOGRAPHY
% 
%%%%%%%%%%%%%%%%%%%%%%%%%%%%%%%%%%%
\bibliographystyle{plainnat}
\begingroup
\raggedright
\emergencystretch 1.5em
\bibliography{../common}
\endgroup

\end{document}
